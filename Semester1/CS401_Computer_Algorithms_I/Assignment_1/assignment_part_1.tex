\documentclass[name=Nihar\ Shailesh\ Joshi, niharid=njoshi27, course=CS 401, num=1 Part 1]{main}

\usepackage{main}

\begin{document}

\problem{1a}

\begin{question}
    True or false? In every instance of the Stable Matching Problem, there is a 
    stable matching containing a pair \textit{(m, w)} such that \textit{m} is 
    ranked first on the preference list of \textit{w} and \textit{w} is ranked 
    first on the preference list of \textit{m}.
    $ $\newline
\end{question}

\begin{solution}
    \textbf{False.}
    Let us consider a group containing 3 men (\textit{m\textsubscript{1}}, 
    \textit{m\textsubscript{2}}, \textit{m\textsubscript{3}}) and 3 women 
    (\textit{w\textsubscript{1}}, \textit{w\textsubscript{2}}, 
    \textit{w\textsubscript{3}}). 
    \\[0.2in]
    Let the preferences of \textit{m\textsubscript{1}}, 
    \textit{m\textsubscript{2}} and \textit{m\textsubscript{3}} be as follows: 
    \begin{center}
        \begin{tabular}{ | c | c | c | c | }
            \hline
            & 1\textsuperscript{st} & 2\textsuperscript{nd} & 3\textsuperscript{rd} \\ \hline
            m\textsubscript{1} & w\textsubscript{2} & w\textsubscript{3} & w\textsubscript{1} \\ \hline
            m\textsubscript{2} & w\textsubscript{2} & w\textsubscript{3} & w\textsubscript{1} \\ \hline
            m\textsubscript{3} & w\textsubscript{1} & w\textsubscript{3} & w\textsubscript{2} \\
            \hline
        \end{tabular}
    \end{center}
    Let the preferences of \textit{w\textsubscript{1}}, 
    \textit{w\textsubscript{2}} and \textit{w\textsubscript{3}} be as follows: 
    \begin{center}
        \begin{tabular}{ | c | c | c | c | }
            \hline
            & 1\textsuperscript{st} & 2\textsuperscript{nd} & 3\textsuperscript{rd} \\ \hline
            w\textsubscript{1} & m\textsubscript{2} & m\textsubscript{1} & m\textsubscript{3} \\ \hline
            w\textsubscript{2} & m\textsubscript{3} & m\textsubscript{2} & m\textsubscript{1} \\ \hline
            w\textsubscript{3} & m\textsubscript{2} & m\textsubscript{1} & m\textsubscript{3} \\
            \hline
        \end{tabular}
    \end{center}
    If we run the stable matching on the above set of men and women, we get the 
    following matches (stable pairs): 
    \begin{center}
        \begin{tabular}{ c c }
            % \hline
            (m\textsubscript{1} & w\textsubscript{3}) \\
            (m\textsubscript{2} & w\textsubscript{2}) \\
            (m\textsubscript{3} & w\textsubscript{1}) \\
            % \hline
        \end{tabular}
    \end{center}
    As we see, in this particular outcome, there is no pair \textit{(m,w)} 
    such that \textit{m} and \textit{w} are ranked first on each other's 
    preference lists. 
    \\[0.2in]
    In fact, no matter which man we start the algorithm from, we do not find 
    the claim to be true.
    \begin{itemize}
        \item For \textit{m\textsubscript{1}} to be paired with his top 
        preference, he would have to be paired with 
        \textit{w\textsubscript{2}}. However, the top preference of 
        \textit{w\textsubscript{2}} is \textit{m\textsubscript{3}}.
        \item Similarly, in the case of \textit{m\textsubscript{2}}, even 
        though his top preference is \textit{w\textsubscript{2}}, the top 
        preference of \textit{w\textsubscript{2}} is 
        \textit{m\textsubscript{3}}.
        \item Lastly, for \textit{m\textsubscript{3}}, even though his top 
        preference is \textit{w\textsubscript{1}}, the top preference of 
        \textit{w\textsubscript{1}} is \textit{m\textsubscript{2}}.
    \end{itemize}
    Therefore, we cannot say that in every instance of the Stable Matching 
    Problem, there is a stable matching containing a pair \textit{(m,w)} such 
    that \textit{m} is ranked first on the preference list of \textit{w} and 
    \textit{w} is ranked first on the preference list of \textit{m}.
\end{solution}

\begin{center}
    \rule{1in}{0.5pt}
\end{center}

\problem{1b}

\begin{question}
    True or false? Consider an instance of the Stable Matching Problem in which 
    there exists a man \textit{m} and a woman \textit{w} such that \textit{m} 
    is ranked first on the preference list of \textit{w} and \textit{w} 
    is ranked first on the preference list of \textit{m}. Then in every stable 
    matching \textit{S} for this instance, the pair \textit{(m, w)} belongs to 
    \textit{S}.
    $ $\newline
\end{question}

\begin{solution}
    \textbf{True.} Proof by contradiction: 
    \\[0.2in]
    Let us consider a group containing 3 men (\textit{m\textsubscript{1}}, 
    \textit{m\textsubscript{2}}, \textit{m\textsubscript{3}}) and 3 women 
    (\textit{w\textsubscript{1}}, \textit{w\textsubscript{2}}, 
    \textit{w\textsubscript{3}}). 
    \\[0.2in]
    Let the preferences of \textit{m\textsubscript{1}}, 
    \textit{m\textsubscript{2}} and \textit{m\textsubscript{3}} be as follows: 
    \begin{center}
        \begin{tabular}{ | c | c | c | c | }
            \hline
            & 1\textsuperscript{st} & 2\textsuperscript{nd} & 3\textsuperscript{rd} \\ \hline
            m\textsubscript{1} & w\textsubscript{2} & w\textsubscript{3} & w\textsubscript{1} \\ \hline
            m\textsubscript{2} & w\textsubscript{3} & w\textsubscript{1} & w\textsubscript{2} \\ \hline
            m\textsubscript{3} & w\textsubscript{1} & w\textsubscript{2} & w\textsubscript{3} \\
            \hline
        \end{tabular}
    \end{center}
    Let the preferences of \textit{w\textsubscript{1}}, 
    \textit{w\textsubscript{2}} and \textit{w\textsubscript{3}} be as follows: 
    \begin{center}
        \begin{tabular}{ | c | c | c | c | }
            \hline
            & 1\textsuperscript{st} & 2\textsuperscript{nd} & 3\textsuperscript{rd} \\ \hline
            w\textsubscript{1} & m\textsubscript{1} & m\textsubscript{2} & m\textsubscript{3} \\ \hline
            w\textsubscript{2} & m\textsubscript{3} & m\textsubscript{1} & m\textsubscript{2} \\ \hline
            w\textsubscript{3} & m\textsubscript{2} & m\textsubscript{1} & m\textsubscript{3} \\
            \hline
        \end{tabular}
    \end{center}
    Here, we see that both \textit{m\textsubscript{2}} and 
    \textit{w\textsubscript{3}} are ranked first on each other's preference 
    lists. 
    \\[0.2in]
    Now, let us assume that there exists a stable matching 
    \textit{S\textsuperscript{'}} such that \textit{m\textsubscript{2}} and 
    \textit{w\textsubscript{3}} are \textbf{not} paired with each other. 
    \\[0.2in]
    We know two facts about the Stable Matching algorithm:
    \begin{itemize}
        \item Men will only propose to women \textbf{in the order of their 
        preference list.} Therefore, in the case of man 
        \textit{m\textsubscript{2}}, proposals will only happen in the order 
        \textit{w\textsubscript{3}} $\rightarrow$ \textit{w\textsubscript{1}} 
        $\rightarrow$ \textit{m\textsubscript{2}}.
        \item Every man will get a chance to propose. Therefore, 
        \textit{m\textsubscript{2}} will surely get a chance to propose to 
        \textit{w\textsubscript{3}} - and as \textit{m\textsubscript{2}} is the 
        top preference of \textit{w\textsubscript{3}}, she will always match 
        with him - as \textbf{women are never left unmatched, they always 
        trade up.}
    \end{itemize}
    As a result, we cannot find a scenario where \textit{m\textsubscript{2}} 
    and \textit{w\textsubscript{3}} do not end up together. \textbf{This fact 
    remains true regardless of which group does the proposals.}
    \\[0.2in]
    Even if, for the sake of the argument, we assume that there does exist a 
    matching where \textit{m\textsubscript{2}} and \textit{w\textsubscript{3}} 
    do not end up together, we can see that they would form an 
    \textbf{unstable pair} - that is, they would both end up in matches where 
    they prefer each other over their current partners.
    \\[0.2in]
    \textbf{This contradicts the definition of Stable Matching.}
    \\[0.2in]
    Therefore, in every stable matching \textit{S} for a man \textit{m} and a 
    woman \textit{w}, such that \textit{m} is ranked first on the preference 
    list of \textit{w} and \textit{w} is ranked first on the preference list of 
    \textit{m}, the pair \textit{(m, w)} belongs to belongs \textit{S}.
\end{solution}

\begin{center}
    \rule{1in}{0.5pt}
\end{center}

\problem{2 - Option b}

\begin{question}
    Give an example of a set of preference lists for which there is a switch 
    that would improve the partner of a woman who switched preferences.
    $ $\newline
\end{question}

\begin{solution}
    Let us consider a group of 3 men \textit{A, B, C} and 3 women 
    \textit{X, Y, Z}.
    \\[0.2in]
    Let their respective preferences be as follows:
    \begin{center}
        \begin{tabular}{ | c | c | c | c | }
            \hline
            & 1\textsuperscript{st} & 2\textsuperscript{nd} & 3\textsuperscript{rd} \\ \hline
            A & X & Y & Z \\ \hline
            B & Y & X & Z \\ \hline
            C & X & Y & Z \\
            \hline
        \end{tabular}
    \end{center}
    \begin{center}
        \begin{tabular}{ | c | c | c | c | }
            \hline
            & 1\textsuperscript{st} & 2\textsuperscript{nd} & 3\textsuperscript{rd} \\ \hline
            X & B & A & C \\ \hline
            Y & A & B & C \\ \hline
            Z & A & B & C \\
            \hline
        \end{tabular}
    \end{center}
    After we run the Gale-Shapely algorithm on the above set of participants, 
    we get the following stable matching: 
    \begin{center}
        \begin{tabular}{ c c }
            % \hline
            (A & X) \\
            (B & Y) \\
            (C & Z) \\
            % \hline
        \end{tabular}
    \end{center}
    If we take a look at woman \textit{X}, we see that her top choice is man 
    \textit{B} but she ends up with man \textit{A}.
    \\[0.2in]
    Now, suppose woman \textit{X} lies about her preferences and says that her 
    preference order is \textbf{\textit{B} $\rightarrow$ \textit{C} 
    $\rightarrow$ \textit{A}} instead. Now we have the new, updated list of 
    preferences as follows:
    \begin{center}
        \begin{tabular}{ | c | c | c | c | }
            \hline
            & 1\textsuperscript{st} & 2\textsuperscript{nd} & 3\textsuperscript{rd} \\ \hline
            A & X & Y & Z \\ \hline
            B & Y & X & Z \\ \hline
            C & X & Y & Z \\
            \hline
        \end{tabular}
    \end{center}
    \begin{center}
        \begin{tabular}{ | c | c | c | c | }
            \hline
            & 1\textsuperscript{st} & 2\textsuperscript{nd} & 3\textsuperscript{rd} \\ \hline
            \textbf{X} & \textbf{B} & \textbf{C} & \textbf{A} \\ \hline
            Y & A & B & C \\ \hline
            Z & A & B & C \\
            \hline
        \end{tabular}
    \end{center}
    When we run the Gale-Shapely algorithm on this new set of preferences, we 
    get the following stable matching:
    \begin{center}
        \begin{tabular}{ c c }
            % \hline
            (A & Y) \\
            (\textbf{B} & \textbf{X}) \\
            (C & Z) \\
            % \hline
        \end{tabular}
    \end{center}
    In this scenario, we see that \textbf{woman \textit{X} actually ends up 
    with man \textit{B}, her first preference.} By lying, woman \textit{X} has 
    actually hoodwinked the algorithm!
    \\[0.2in]
    Therefore, this is an example where a switch in preferences improves the 
    partner of a woman.
    \\[0.2in]
    Consequently, we can also conclude that \textbf{yes, a man or a woman can 
    end up better off by lying about his or her preferences.}
\end{solution}

\begin{center}
    \rule{1in}{0.5pt}
\end{center}

\end{document}